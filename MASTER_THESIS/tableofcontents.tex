\chapter{WYKAZ WAŻNIEJSZYCH OZNACZEŃ I SKRÓTÓW}
\label{chpt:wykaz-ważniejszych-oznaczeń-i-skrótów}



\chapter{WSTĘP I CEL PRACY}
\label{chpt:wstęp-i-cel-pracy}



\chapter{WSTĘP TEORETYCZNY}
\label{chpt:wstęp-teoretyczny}
\section{Zastosowania światła w medycynie}
\subsection{Fototerapia}
\section{Metody symulacji światła}
\subsection{Metody analityczne}
\subsection{Metody statystyczne}
\subsection{Kryteria wyboru}
\section{Metoda Monte Carlo}
\subsection{Symulowanie propagacji światła}
\section{Przykładowe implementacje symulacji światła}
\section{Algorytm maszerujących sześcianów}



\chapter{DIAGRAM APLIKACJI}
\label{chpt:diagram-aplikacji}



\chapter{ZARZĄDZANIE GENERATORAMI LICZB PSEUDOLOSOWYCH}
\label{chpt:zarządzanie-generatorami-liczb-pseudolosowych}
\section{Interfejs generatorów}
\section{Interfejs funkcji losujących}
\subsection{Wykorzystane funkcje losujące}
- (funkcja gęstości prawdopodobieństwa, dystrybuanta, funkcja próbkująca cechy)
\subsection{Alternatywne funkcje losujące}

- precyzja
- nieużywana w większości przypadków
- możliwa do wykorzystania przy użyciu alternatywnej funkcji losującej
- zamiast losować z liczby z rozkładu jednostajnego [low,high), losuje liczby całkowite z przedziału [0,(high-low)*10^prec], a następnie dzieli je przez 10^prec i dodaje low. Ma to zapewnić przedział obustronnie zamknięty wylosowanych liczb.



\chapter{IMPLEMENTACJA TRÓJWYMIAROWYCH MODELI SKÓRY}
\label{chpt:implementacja-trójwymiarowych-modeli-skóry}
\section{Jednolity ośrodek propagacji}
\section{Wielowarstowe modele ośrodka propagacji}
\subsection{Implementacja bazująca na trójwymiarowej tablicy}
\subsubsection{Identyfikacja przekroczenia granicy ośrodków}
\subsubsection{Modele skóry}
\paragraph{Dwuwarstwowy model skóry}
\paragraph{Wielowarstwowy model skóry}
\paragraph{Ziarnisty model skóry}
\subsection{Implementacja bazująca na opisie brył geometrycznych}
\subsubsection{Model materiału}
\subsubsection{Zakładkowanie warstw}
\subsubsection{Identyfikacja przekroczenia granicy ośrodków}
\subsubsection{Modele skóry}
\paragraph{Dwuwarstwowy model skóry}
\paragraph{Wielowarstwowy model skóry}
\section{Zapis do pliku}



\chapter{IMPLEMENTACJA ŹRÓDŁA ŚWIATŁA}
\label{chpt:implementacja-źródła-światła}
\section{Agregowanie punktowych źródeł światła}
\section{Punktowe źródło światła}
\section{Wiązka fotonów}
\section{Zapis do pliku}
\section{Porównanie wpływu liczby wiązek fotonów na wynik symulacji}



\chapter{AGREGACJA ELEMENTÓW STANU POCZĄTKOWEGO}
\label{chpt:agregacja-elementów-stanu-początkowego}
\section{Zapis do pliku}



\chapter{PRZECHOWYWANIE WYNIKÓW}
\label{chpt:przechowywanie-wyników}
\section{Rejestr wiązek fotonów}
\section{Rejestr pozycji fotonów}
\section{Tablica absorpcji fotonów}



\chapter{OBLICZENIA NUMERYCZNE PROPAGACJI ŚWIATŁA}
\label{chpt:obliczenia-numeryczne-propagacji-światła}
\section{Plik konfiguracyjny}
- flagi tworzenia nowych plików domyślnych elementów stanu początkowego środowiska propagacji
- alternatywne ścieżki do plików z elementami stanu początkowego środowiska propagacji
- flagi wyboru plików domyślnych lub wczytania plików z wykorzystaniem ścieżek alternatywnych
- flagi wyboru sposobu wczytania stanu początkowego środowiska
+    - z wcześniej wybranych ścieżek
+    - z ścieżek zapisanych w pliku agregującym śrdodowisko propagacji
- Model opisu ośrodka symulacji
+    - opis przez tablicę / opis przez stos brył geometrycznych
- Ignorowanie ośrodka propagacji
+    - globalny materiał ośrodka symulacji
- Probabilistyczny podział wiązki
- Rekurencja przy podziale wiązki
- Zapis rejestru pozycji fotonów w liczbach całkowitych
- włączenie i wyłączenie zapisów do tablicy absorpcji oraz rejestru pozycji fotonów
- ziarno generatora liczb losowych
- precyzja
+    - możliwa do wykorzystania przy generowaniu liczb losowych
- rozdzielczość przestrzenna - liczba przedziałów ośrodka propagacji i tablicy absorpcji przypadająca na 1 cm
- Wpółczynnik anizotropii
- Szansa na ocalenie fotonu w ruletce
- parametry materiałów

\subsection{Porównanie ważniejszych ustawień}
\subsubsection{Model opisu ośrodka symulacji}
- opis przez tablicę / opis przez stos brył geometrycznych
\subsubsection{Probabilistyczny podział wiązki}
\paragraph{Przypadek utknięcia w martwym punkcie}
\subsubsection{Zapis rejestru pozycji fotonów w liczbach całkowitych}
\subsubsection{Wpółczynnik anizotropii}

\section{Zarządzanie uruchamianiem symulacji oraz wyświetlaniem wyników}
- Wczytywanie stanu symulacji
- podstawowe ustawienia wyświetlania wyników
+    - Wczytywanie triangularyzowanych powierzchni tkanek

\section{Tworzenie plików determinujących stan początkowy}

\section{Przebieg symulacji}
\subsection{Emisja wiązki światła}
\subsection{Propagacja}
\subsubsection{Losowanie długości skoku}
\subsubsection{Przesunięcie}
\paragraph{Test przekroczenia granicy ośrodków}
\paragraph{Test przekroczenia granicy obszaru obserwacji}
\subsubsection{Absorpcja}
\subsubsection{Rozproszenie}
\subsubsection{Test zakończenia propagacji metodą ruletki}

\section{Zapis do plików}
\subsection{Stan symulacji}
\subsection{Wyniki propagacji światła}
\section{Normalizacja wyników}
\subsection{Normalizacja tablicy absorpcji}
\subsection{Normalizacja rejestru pozycji fotonów}
\subsection{Sprawdzenie spójności otrzymanych wartości}



\chapter{WIZUALIZACJA WYNIKÓW}
\label{chpt:wizualizacja-wyników}
\section{Narzędzia pomocnicze}
\subsection{Strzałki zmiany pozycji fotonów}
\subsection{Triangularyzowane powierzchnie tkanek}
\subsection{Uniwersalne schematy kolorowania}
\subsubsection{Tworzenie schematów z innych obiektów}
\paragraph{Tablice trójwymiarowe}
\paragraph{Tablice dwuwymiarowe}
\paragraph{Rejestr pozycji fotonów}
\subsubsection{Metody kolorowania}
\paragraph{Progowanie ,,threshold''}
\paragraph{Pętla kolorów ,,loop''}
\subparagraph{Tworzenie palety kolorów}
\paragraph{Jednolity kolor ,,solid''}
\paragraph{Według id wiązki ,,photonwise''}
\paragraph{Losowo ,,random''}
\paragraph{Tęcza ,,rainbow''}
\paragraph{Normalizacja min-max ,,min-max''}
\paragraph{Wyśrodkowanie mediany ,,median''}
\paragraph{Transformacja danych do rozkładu normalnego ,,trans-normal''}
\paragraph{Skala logarytmiczna ,,logarithmic''}
\paragraph{Mapa ciepła normalizacji min-max ,,heatmap min-max''}
\paragraph{Mapa ciepła po wyśrodkowaniu mediany ,,heatmap median''}
\paragraph{Mapa ciepła rozkładu normalnego ,,heatmap trans-normal''}
\paragraph{Mapa ciepła skali logarytmicznej ,,heatmap logarithmic''}
\subsubsection{Narzędzia pomocnicze schematów kolorowania}
\paragraph{Resetowanie kolorów}
\paragraph{Stos schematów kolorowania}
\paragraph{Szukanie przodków wiązki}
- Kolorowanie według najstarszego przodka
- Dodawanie wiązek pokrewnych do wyświetlenia 
\paragraph{Szukanie potomków wiązki}
\paragraph{Sumowanie wartości leżących na zbliżonych pozycjach}
\paragraph{Rozproszenie wartości}

\subsection{Przygotowanie do zapisu obrazów 2D}
\subsubsection{Sumowanie wartości wzdłóż wybranej osi}
\subsubsection{Projekcje obiektów 3D na płaszczyznę 2D}
\paragraph{Projekcje tablic 3D}
\paragraph{Projekcje rejestru pozycji fotonów}
\subsubsection{Plastry tablicy 3D}
\subsection{Zapis do obrazów 2D}
\subsubsection{Rozproszenie pozycji fotonów}

\section{Wykorzystanie biblioteki Matplotlib}
\subsection{Proste wyświetlanie trójwymiarowej tablicy}
\subsection{Próbkowanie trójwymiarowej tablicy}
\subsection{Mapy ciepła}

\section{Wykorzystanie biblioteki Vispy}
\subsection{Proste wyświetlanie trójwymiarowej tablicy}
\subsection{Wizualizacja objętości biblioteki Vispy}
\subsection{Wizualizacja obiektów z wykorzystaniem biblioteki Pandas}

\section{Zarządzanie wizualizacją wyników}
\subsection{Wyświetlanie statystyk}
\subsection{Przygotowanie schematów rysowania strzałek dla tablic}
\subsection{Przygotowanie triangularyzowanych powierzchni tkanek}
\subsection{Przygotowanie stosu brył geometrycznych materiałów}
\subsection{Podgląd środowiska symulacji}
\subsection{Podgląd absorpcji na tle ośrodka propagacji}
\subsection{Wyświetlanie tablicy absorpcji}
\subsection{Wyświetlanie rejestru pozycji fotonów}
\subsection{Wyświetlanie tablicy absorpcji zsumowanej wzdłóż wybranej osi}
\subsection{Projekcje rejestru fotonów na płaszczyznę 2D}



\chapter{ANALIZA I DYSKUSJA UZYSKANYCH WYNIKÓW}
\label{chpt:analiza-i-dyskusja-uzyskanych-wyników}

\section{Przeprowadzone testy}

Test A = Jednolity model skóry
Test B = Dwuwarstwowy model skóry
Test C = Wielowarstwowy model skóry
Test D = Ziarnisty model skóry

\subsection{Jednolity model skóry}
\subsubsection{Własna implementacja}
\paragraph{Nastawy pliku konfiguracyjnego}
\paragraph{Otrzymane wyniki}
\subsubsection{Przykładowa implementacja z literatury}
\subsubsection{Porównanie wyników}
\subsubsection{Wnioski}

\subsection{Dwuwarstwowy model skóry}
\subsubsection{Własna implementacja}
\paragraph{Nastawy pliku konfiguracyjnego}
\paragraph{Otrzymane wyniki}
\subsubsection{Przykładowa implementacja z literatury}
\subsubsection{Porównanie wyników}
\subsubsection{Wnioski}

\subsection{Wielowarstwowy model skóry}
\subsubsection{Własna implementacja}
\paragraph{Nastawy pliku konfiguracyjnego}
\paragraph{Otrzymane wyniki}
\subsubsection{Przykładowa implementacja z literatury}
\subsubsection{Porównanie wyników}
\subsubsection{Wnioski}

\subsection{Ziarnisty model skóry}
\subsubsection{Własna implementacja}
\paragraph{Nastawy pliku konfiguracyjnego}
\paragraph{Otrzymane wyniki}
\subsubsection{Przykładowa implementacja z literatury}
\subsubsection{Porównanie wyników}
\subsubsection{Wnioski}



\chapter{PODSUMOWANIE}
\label{chpt:podsumowanie}



\chapter{WYKAZ LITERATUR}
\label{chpt:wykaz-literatur}
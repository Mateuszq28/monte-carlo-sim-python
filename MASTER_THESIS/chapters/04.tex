\chapter{ZARZĄDZANIE GENERATORAMI LICZB PSEUDOLOSOWYCH}
\label{chpt:zarządzanie-generatorami-liczb-pseudolosowych}
\section{Interfejs generatorów}
\section{Interfejs funkcji losujących}
\subsection{Wykorzystane funkcje losujące}
- (funkcja gęstości prawdopodobieństwa, dystrybuanta, funkcja próbkująca cechy)
\subsection{Alternatywne funkcje losujące}

- precyzja
- nieużywana w większości przypadków
- możliwa do wykorzystania przy użyciu alternatywnej funkcji losującej
- zamiast losować z liczby z rozkładu jednostajnego [low,high), losuje liczby całkowite z przedziału [0,(high-low)*10^prec], a następnie dzieli je przez 10^prec i dodaje low. Ma to zapewnić przedział obustronnie zamknięty wylosowanych liczb.

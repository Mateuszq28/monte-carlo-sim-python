\chapter{OBLICZENIA NUMERYCZNE PROPAGACJI ŚWIATŁA}
\label{chpt:obliczenia-numeryczne-propagacji-światła}
\section{Plik konfiguracyjny}
- flagi tworzenia nowych plików domyślnych elementów stanu początkowego środowiska propagacji
- alternatywne ścieżki do plików z elementami stanu początkowego środowiska propagacji
- flagi wyboru plików domyślnych lub wczytania plików z wykorzystaniem ścieżek alternatywnych
- flagi wyboru sposobu wczytania stanu początkowego środowiska
+    - z wcześniej wybranych ścieżek
+    - z ścieżek zapisanych w pliku agregującym śrdodowisko propagacji
- Model opisu ośrodka symulacji
+    - opis przez tablicę / opis przez stos brył geometrycznych
- Ignorowanie ośrodka propagacji
+    - globalny materiał ośrodka symulacji
- Probabilistyczny podział wiązki
- Rekurencja przy podziale wiązki
- Zapis rejestru pozycji fotonów w liczbach całkowitych
- włączenie i wyłączenie zapisów do tablicy absorpcji oraz rejestru pozycji fotonów
- ziarno generatora liczb losowych
- precyzja
+    - możliwa do wykorzystania przy generowaniu liczb losowych
- rozdzielczość przestrzenna - liczba przedziałów ośrodka propagacji i tablicy absorpcji przypadająca na 1 cm
- Wpółczynnik anizotropii
- Szansa na ocalenie fotonu w ruletce
- parametry materiałów

\subsection{Porównanie ważniejszych ustawień}
\subsubsection{Model opisu ośrodka symulacji}
- opis przez tablicę / opis przez stos brył geometrycznych
\subsubsection{Probabilistyczny podział wiązki}
\paragraph{Przypadek utknięcia w martwym punkcie}
\subsubsection{Zapis rejestru pozycji fotonów w liczbach całkowitych}
\subsubsection{Wpółczynnik anizotropii}

\section{Zarządzanie uruchamianiem symulacji oraz wyświetlaniem wyników}
- Wczytywanie stanu symulacji
- podstawowe ustawienia wyświetlania wyników
+    - Wczytywanie triangularyzowanych powierzchni tkanek

\section{Tworzenie plików determinujących stan początkowy}

\section{Przebieg symulacji}
\subsection{Emisja wiązki światła}
\subsection{Propagacja}
\subsubsection{Losowanie długości skoku}
\subsubsection{Przesunięcie}
\paragraph{Test przekroczenia granicy ośrodków}
\paragraph{Test przekroczenia granicy obszaru obserwacji}
\subsubsection{Absorpcja}
\subsubsection{Rozproszenie}
\subsubsection{Test zakończenia propagacji metodą ruletki}

\section{Zapis do plików}
\subsection{Stan symulacji}
\subsection{Wyniki propagacji światła}
\section{Normalizacja wyników}
\subsection{Normalizacja tablicy absorpcji}
\subsection{Normalizacja rejestru pozycji fotonów}
\subsection{Sprawdzenie spójności otrzymanych wartości}

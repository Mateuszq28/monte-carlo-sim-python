\chapter{WIZUALIZACJA WYNIKÓW}
\label{chpt:wizualizacja-wyników}
\section{Narzędzia pomocnicze}
\subsection{Strzałki zmiany pozycji fotonów}
\subsection{Triangularyzowane powierzchnie tkanek}
\subsection{Uniwersalne schematy kolorowania}
\subsubsection{Tworzenie schematów z innych obiektów}
\paragraph{Tablice trójwymiarowe}
\paragraph{Tablice dwuwymiarowe}
\paragraph{Rejestr pozycji fotonów}
\subsubsection{Metody kolorowania}
\paragraph{Progowanie ,,threshold''}
\paragraph{Pętla kolorów ,,loop''}
\subparagraph{Tworzenie palety kolorów}
\paragraph{Jednolity kolor ,,solid''}
\paragraph{Według id wiązki ,,photonwise''}
\paragraph{Losowo ,,random''}
\paragraph{Tęcza ,,rainbow''}
\paragraph{Normalizacja min-max ,,min-max''}
\paragraph{Wyśrodkowanie mediany ,,median''}
\paragraph{Transformacja danych do rozkładu normalnego ,,trans-normal''}
\paragraph{Skala logarytmiczna ,,logarithmic''}
\paragraph{Mapa ciepła normalizacji min-max ,,heatmap min-max''}
\paragraph{Mapa ciepła po wyśrodkowaniu mediany ,,heatmap median''}
\paragraph{Mapa ciepła rozkładu normalnego ,,heatmap trans-normal''}
\paragraph{Mapa ciepła skali logarytmicznej ,,heatmap logarithmic''}
\subsubsection{Narzędzia pomocnicze schematów kolorowania}
\paragraph{Resetowanie kolorów}
\paragraph{Stos schematów kolorowania}
\paragraph{Szukanie przodków wiązki}
- Kolorowanie według najstarszego przodka
- Dodawanie wiązek pokrewnych do wyświetlenia 
\paragraph{Szukanie potomków wiązki}
\paragraph{Sumowanie wartości leżących na zbliżonych pozycjach}
\paragraph{Rozproszenie wartości}

\subsection{Przygotowanie do zapisu obrazów 2D}
\subsubsection{Sumowanie wartości wzdłóż wybranej osi}
\subsubsection{Projekcje obiektów 3D na płaszczyznę 2D}
\paragraph{Projekcje tablic 3D}
\paragraph{Projekcje rejestru pozycji fotonów}
\subsubsection{Plastry tablicy 3D}
\subsection{Zapis do obrazów 2D}
\subsubsection{Rozproszenie pozycji fotonów}

\section{Wykorzystanie biblioteki Matplotlib}
\subsection{Proste wyświetlanie trójwymiarowej tablicy}
\subsection{Próbkowanie trójwymiarowej tablicy}
\subsection{Mapy ciepła}

\section{Wykorzystanie biblioteki Vispy}
\subsection{Proste wyświetlanie trójwymiarowej tablicy}
\subsection{Wizualizacja objętości biblioteki Vispy}
\subsection{Wizualizacja obiektów z wykorzystaniem biblioteki Pandas}

\section{Zarządzanie wizualizacją wyników}
\subsection{Wyświetlanie statystyk}
\subsection{Przygotowanie schematów rysowania strzałek dla tablic}
\subsection{Przygotowanie triangularyzowanych powierzchni tkanek}
\subsection{Przygotowanie stosu brył geometrycznych materiałów}
\subsection{Podgląd środowiska symulacji}
\subsection{Podgląd absorpcji na tle ośrodka propagacji}
\subsection{Wyświetlanie tablicy absorpcji}
\subsection{Wyświetlanie rejestru pozycji fotonów}
\subsection{Wyświetlanie tablicy absorpcji zsumowanej wzdłóż wybranej osi}
\subsection{Projekcje rejestru fotonów na płaszczyznę 2D}
